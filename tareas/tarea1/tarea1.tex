\documentclass[11pt,letterpaper]{article}
\usepackage[utf8]{inputenc}
\usepackage[letterpaper,includeheadfoot, top=0.5cm, bottom=3.0cm, right=1.5cm, left=1.5cm]{geometry}
\renewcommand{\familydefault}{\sfdefault}
\usepackage{float} % Allows putting an [H] in \begin{figure} to specify the exact location of the figure
\renewcommand{\figurename}{Fig.}
\usepackage{lmodern}% http://ctan.org/pkg/lm
\usepackage{amsmath}
\usepackage{graphicx}
\usepackage{color}
\usepackage{hyperref}
\usepackage{amssymb}
\usepackage{url}
\usepackage{pdfpages}
\usepackage{fancyhdr}
\usepackage{subfig}
\usepackage{listings} %Codigo
\usepackage{selinput}                   % Compatibilidad con acentos
\newcommand{\bR}{\ensuremath{\mathbb{R}}}
\newcommand{\bN}{\ensuremath{\mathbb{N}}}
\newcommand{\bZ}{\ensuremath{\mathbb{Z}}}
\newcommand{\bP}{\ensuremath{\mathbb{P}}}
\newcommand{\bE}{\ensuremath{\mathbb{E}}}
\newcommand{\bD}{\ensuremath{\mathbb{D}}}
\newcommand{\bV}{\ensuremath{\mathbb{V}}}
\newcommand{\cN}{\ensuremath{\mathcal{N}}}
\newcommand{\x}{\ensuremath{\mathbf{x}}}
\newcommand{\m}{\ensuremath{\mathbf{m}}}
\newboolean{pauta}
\setboolean{pauta}{false}
%\usepackage{tkz-graph}
%\usetikzlibrary{arrows}
%\usepackage{algorithm}
%\usepackage{algorithmic}

\lstset{language=C, tabsize=4,framexleftmargin=5mm,breaklines=true}

\begin{document}

% ·············· ENCABEZADO - PIE DE PAGINA ············
\pagestyle{fancy}
\fancyhf{}
\lhead{\textbf{MA5204: Aprendizaje de Máquinas 2019}}
\rfoot{Page \thepage}
%Encabezado

% =============== Inicio Documento ===============
%\rm
\headheight = 14pt
\begin{center}
\large {\textbf{Tarea 1}}\\
\end{center}
\textbf{Profesor:} Felipe Tobar\\ 
\textbf{Auxiliares:} Mauricio Araneda, Alejandro Cuevas, Mauricio Romero \\
\textbf{Consultas:} Mauricio Araneda \\
\textbf{Fecha entrega:} 10/3/2019 \\

\noindent\textbf{Formato entrega:} Informe en formato PDF, con una extensión máxima de 3 páginas (puede usar un formato de doble columna), presentando y analizando sus resultados, y detallando la metodología utilizada. Adicionalmente debe entregar el jupyter notebook (o el código que haya generado) con la resolución de la tarea.

\vspace{5mm}

\noindent\textbf{P1. Máxima Verosimilitud} \textbf{(2.0 puntos, tomado de MacKay p.309)}
\vspace{5 mm}

Siete científicos con habilidades experimentales salvajemente dispares reportan distintas estimaciones de un parámetro $\mu$

\begin{table}[H]
\centering
\begin{tabular}{llllllll}
Científico & A       & B    & C     & D     & E     & F     & G      \\ \hline
Estimación & -27.020 & 3.57 & 8.191 & 9.898 & 9.603 & 9.945 & 10.056
\end{tabular}
\end{table}

Discuta cómo encontrar el parámetro óptimo y cuán confiable es cada científico. Para
este fin asuma que la estimación de cada científico puede ser considerada como una muestra de distribuciones normales de igual media ($\mu$) pero distintas varianzas ($\sigma_{1}^2 , \dots , \sigma_{7}^2 )$. Observe de los datos que las estimaciones de los científicos A y B son poco confiables y que el parámetro buscado debería estar entre 9 y 10 ¿es posible abordar este problema usando máxima verosimilitud?

Valide (o rechace) su respuesta mediante simulaciones. Explicite sus supuestos e interprete sus resultados. (Hint: Grafique la verosimilitud)


\vspace{5 mm}
\noindent\textbf{P2. Regresión Lineal} \textbf{(3.5 puntos)}
\vspace{5 mm}

En el archivo \textbf{datos/szege\_clima.csv} se encuentra un \emph{dump} (subconjunto del  elementos del dataset original) de datos climáticos de la ciudad de Szege, la tercera ciudad más grande de Hungría. El archivo consiste de dos columnas, la primera $X$ corresponde a la proporcion de humedad ambiental, la segunda $Y$ corresponde a la sensación térmica. Mayor detalle sobre los datos utilizados puede encontrarse en:

\vspace{1em}
\centerline{\url{https://www.kaggle.com/budincsevity/szeged-weather}}
\vspace{1em}

Para este conjunto de datos se pide implementar una regresión lineal regularizada en Python donde $X$ será el regresor y $Y$ la variable a estimar. Esto usando solo operaciones álgebra lineal.
Para esto deberá:
\begin{itemize}
	\item[(0)] Instalar la última versión de Anaconda y lanzar Jyputer Notebook.
	\item[(a)] (0.5 puntos) Cargar los datos desde el archivo \textbf{datos/szege\_clima.csv} y graficarlos.
	\item[(b)] (0.5 puntos) Crear y ajustar su modelo de regresión lineal regularizada, implementando regularización \emph{ridge}, pruebe con distintos valores del factor $\rho$ (partiendo de $\rho=0$).
	\item[(c)] (0.5 puntos) Crear y ajustar su modelo de regresión lineal regularizada, implementando regularización \emph{lasso}.
	\item[(d)] (0.5 puntos) Generar una predicción para 10 puntos de forma uniforme en el intervalo [0, 1] para ambos modelos.
	\item[(e)] (1.5 puntos) Grafique y discuta los resultados. A modo de guía en la discusión puede considerar las siguientes preguntas (no se limite únicamente a éstas):
	    \begin{itemize}
	        \item ¿En qué afecta que las regresiones estén regularizadas sobre este conjunto de datos?
	        \item ¿Qué pros y contra otorga cada regularización? ¿Cuál es preferible en este contexto?
	        \item ¿Qué modelo tiene mejor desempeño en sus predicciones? ¿Cuál tiene el menor error asociado?
	    \end{itemize}
\end{itemize}


 \textbf{No} se permite el uso de paquetes predefinidos para regresión lineal. Estos pueden ser considerados para contrastar los propios resultados pero no para resolver la pregunta. E.g., \texttt{numpy.polyfit},  \texttt{scipy.stats.linregress}, \texttt{sklearn.linear\_model.LinearRegression}

\vspace{5 mm}
\noindent\textbf{P3. Proyecto curso} \textbf{(0.5 puntos)}
\vspace{5 mm}

Discuta brevemente su idea de proyecto para el curso (50-100 palabras). El proyecto puede ser aplicado o más teórico, motivado por problemas existentes de sus proyectos/memorias/tesis, emprendimientos, o bien puede inspirarse con los proyectos de años anteriores en:

\vspace{1em}
\centerline{\url{http://games.cmm.uchile.cl/courses/MA5203/}}
\vspace{1em}


\vspace{2em}





\end{document}

