\documentclass[11pt,letterpaper]{article}
\usepackage[utf8]{inputenc}
\usepackage[letterpaper,includeheadfoot, top=0.5cm, bottom=3.0cm, right=1.5cm, left=1.5cm]{geometry}
\renewcommand{\familydefault}{\sfdefault}
\usepackage{float} % Allows putting an [H] in \begin{figure} to specify the exact location of the figure
\renewcommand{\figurename}{Fig.}
\usepackage{lmodern}% http://ctan.org/pkg/lm
\usepackage{amsmath}
\usepackage{graphicx}
\usepackage{color}
\usepackage{hyperref}
\usepackage{amssymb}
\usepackage{url}
\usepackage{pdfpages}
\usepackage{fancyhdr}
\usepackage{subfig}
\usepackage{listings} %Codigo
\usepackage{selinput}                   % Compatibilidad con acentos
\newcommand{\bR}{\ensuremath{\mathbb{R}}}
\newcommand{\bN}{\ensuremath{\mathbb{N}}}
\newcommand{\bZ}{\ensuremath{\mathbb{Z}}}
\newcommand{\bP}{\ensuremath{\mathbb{P}}}
\newcommand{\bE}{\ensuremath{\mathbb{E}}}
\newcommand{\bD}{\ensuremath{\mathbb{D}}}
\newcommand{\bV}{\ensuremath{\mathbb{V}}}
\newcommand{\cN}{\ensuremath{\mathcal{N}}}
\newcommand{\x}{\ensuremath{\mathbf{x}}}
\newcommand{\m}{\ensuremath{\mathbf{m}}}
\newboolean{pauta}
\setboolean{pauta}{false}
%\usepackage{tkz-graph}
%\usetikzlibrary{arrows}
%\usepackage{algorithm}
%\usepackage{algorithmic}

\lstset{language=C, tabsize=4,framexleftmargin=5mm,breaklines=true}

\begin{document}

% ·············· ENCABEZADO - PIE DE PAGINA ············
\pagestyle{fancy}
\fancyhf{}
\lhead{\textbf{MA5204: Aprendizaje de Máquinas 2019}}
\rfoot{Page \thepage}
%Encabezado

% =============== Inicio Documento ===============
%\rm
\headheight = 14pt
\begin{center}
\large {\textbf{Tarea 4: Máquinas de soporte vectorial}}\\
\end{center}
\textbf{Profesor:} Felipe Tobar\\ 
\textbf{Auxiliares:} Mauricio Araneda, Alejandro Cuevas, Mauricio Romero \\
\textbf{Consultas:} Todo el cuerpo docente.\\
\textbf{Fecha entrega:} 23/6/2019 \\
\textbf{Formato entrega:} Entregue un informe en formato PDF con una extensión de a lo más 2 páginas presentando y analizando sus resultados, detalle la metodología utilizada y adicionalmente debe entregar un jupyter notebook con los códigos que creó para resolver la tarea.
% \vspace{5mm}

\begin{center}
    \textit{Esta es una de las 2 versiones de la tarea4, deben elegir solo una de las versiones (SVM o NN).}
\end{center}


\noindent\textbf{P1. Clasificación con SVM}
\vspace{5mm}

Se pretende realizar la tarea de clasificación sobre el dataset MNIST, usando Support Vector Machines (SVM) como clasificador, para esto:

\begin{enumerate}

\item[a)](1 pts.) Cargue los datos (Vea el demo para cargarlos) y sepárelos de forma aleatoria en conjuntos de entrenamiento $5/7$, validación $1/7$ y prueba $1/7$(si carga con pytorch el conjunto de prueba ya esta definido).

Normalice de forma estándar los datos de modo que tengan media $0$ y varianza unitaria, es decir, $\x_{\text{new}} = \frac{\x - \mu}{\sigma}$ donde ($\mu$, $\sigma$) es la media y desviación estándar del conjunto de entrenamiento por dimensión.

\textit{En caso de cargar los datos con Pytorch:} Transforme sus datos de tal forma que cada imagen de $28 \times 28$ sea un vector de tamaño $784$.

\textbf{Observación:} Le puede ser util train\_test\_split y StandardScaler de sklearn.

\item [b) ](2 pts.) Visualice el conjunto de datos sobre un grafico de dos dimensiones para el conjunto de entrenamiento mediante PCA. Debe marcar con un color distinto los elementos dependiendo de la clase a la que corresponden. ¿Se distinguen tendencias para la separación del conjunto en sus clases? ¿Qué clases son más fáciles de distinguir de otras?

\item[c) ](3 pts.) Implemente tres funciones de kernel a utilizar en SVM, una de ellas debe ser el kernel Gaussiano o RBF, las otras 2 son a elección.

Para esto implemente una función que tome como argumentos los conjuntos de samples $X_1 y X_2$ y retorne una matriz de $n_{samples 1}$ x $n_{samples 2}$ con la evaluación del kernel, es decir el elemento $(i, j)$ de dicha matriz es igual a $k(X_1(i), X_2(j))$ la evaluación del kernel del elemento $i$ con el elemento $j$.


\item[d) ](3 pts.) Entrene un clasificador SVM con 5 distintos valores de $C$( $\star$) usando su kernel RBF implementado. Muestre los errores totales (i.e., datos mal clasificados) que comete su clasificador SVM en función de distintos valores de $C$ , utilizando un kernel de su elección), tanto para los datos de entrenamiento como los de validación. Vea para qué valor de $C$ se obtiene el mínimo número de errores tanto en los datos de entrenamiento y validación, luego evalue en el conjunto de test. Comente sobre las diferencias de los resultados obtenidos.

\item[e) ](3 pts.) Entrene un SVM con kernel los tres kernel previamente definidos (NO puede usar los kernels predefinidos de SVC), considerando el mejor $C$ encontrado en la parte d). Analice sus resultados comparando el desempeño de los distintos kernels mediante la \textit{accuracy} y matríz de confusión sobre el conjunto de validación y test. Discuta sus resultados.

\end{enumerate}

\textbf{Observaciones}: Recomendaciones generales para valores de parámetros y documentación extra:

\begin{itemize}
    \item $\star$ Le puede ser util la documentación de sklearn \url{https://scikit-learn.org/stable/modules/generated/sklearn.svm.SVC.html}.
    \item En la parte \textbf{c)} le puede ser util considerar alguno de los kernels definidos en el siguiente enlace: \url{http://crsouza.com/2010/03/17/kernel-functions-for-machine-learning-applications/}
    \item En la parte \textbf{c)} En su implementación \textbf{evite hacer uso de ciclos for}, aproveche la notación matricial, de otro modo su código puede ser muy ineficiente, para generar la matriz de distancias le puede servir \url{https://docs.scipy.org/doc/scipy/reference/generated/scipy.spatial.distance_matrix.html}.
    \item El tiempo de entrenamiento para el clasificador debería tomar al rededor de 2 a 6 minutos cada vez que se instancie. 
    \item En la parte \textbf{d)} considere $C \in [0,1]$.
\end{itemize}



\end{document}
